\documentclass[onecolumn, draftclsnofoot,10pt, compsoc]{IEEEtran}
\usepackage{graphicx}
\usepackage{url}
\usepackage{setspace}

\usepackage{geometry}
\geometry{textheight=9.5in, textwidth=7in}
\usepackage{hyperref}

\def \DocType{		WriteUp
				}
			
\newcommand{\NameSigPair}[1]{\par
\makebox[2.75in][r]{#1} \hfil 	\makebox[3.25in]{\makebox[2.25in]{\hrulefill} \hfill		\makebox[.75in]{\hrulefill}}
\par\vspace{-12pt} \textit{\tiny\noindent
\makebox[2.75in]{} \hfil		\makebox[3.25in]{\makebox[2.25in][r]{Signature} \hfill	\makebox[.75in][r]{Date}}}}
% 3. If the document is not to be signed, uncomment the RENEWcommand below
\renewcommand{\NameSigPair}[1]{#1}




%group members: Nipun Bathini, Parker Bruni
%group number: 20




\begin{document}

\pagenumbering{arabic}


\section{Design Plan}

		Our plan of action is to research the slob.c first fit implementation, research the best fit algorithm, view the slob.c implementation in our kernel and find a template slob.c 
		first fit algorithm that can be implemented into our kernel. We will edit the current implementation to include the best fit algorithm that allocates space not by choosing
		the first available free block of space in memory but to choose the one that fits a block that leaves the smalles amount of free space. We will record the necessary changes to that 
		file and other files in a patch file. We will then look into how to generate a report of fragmentation by figuring out how to return the memory usage of our system under the 
		first fit algorithm versus the best fit algorithm.
	
\section{Questions}
	\begin{enumerate}
		\item \textbf{What do you think the main point of this assignment is?}
	
		We believe that the point of this assignment was to help us understand a linux kernel
		environment enough to create modify the currently existing slob allocator algorithm to a more efficient algorithm (best fit).
		We also will learn the differences between the algorithms and learn about memory allocation and fragmentation. 
		
		\item \textbf{How did you personally approach the problem? Design decisions, algorithm, etc.}
	
		We studied online about linux kernel allocators and version of allocators and chose a method 
		to implement the best fit algorithm instead of the current first fit algorithm that is in place.
		Our algorithm will analyse all free blocks of memory and choose an allocation based on which block 
		will leave the smallest amount of space between the new allocated memory and existing allocated memory.
		
		\item \textbf{How did you ensure your solution was correct? Testing details, for instance.}
			
		To ensure the solution was correct we utilized syscalls to display to us the amount of free space and used space for each algorithm before and after the patch.
		These values can be compared to determine efficiency, in this case best fit was more efficient as expected.

		
		\item \textbf{What did you learn?}
	
		We learned more about linux memory allocation and related algorithms, memory fragmentation, and terms used to describe
		things relating to memory. We also learned how to utilize syscalls to view the systems data.
		
		\item \textbf{Hpw should the TA test your patch?}
		
		View the differences in memory usage due to changes in fragmentation caused by better fit memory allocation. 
		This comes from our syscall functions that we implemented "sys_slob_space_free(void)" and "sys_slob_space_used(void)" which are utilized in our test program test.c.
		test.c will print the total space free and the total space used, these values can be compared to see which was more efficient.
		
		run test.c before patching and after patching and compare the results to determine which was more efficient.
		
	\end{enumerate}

\section{Version Control Log}

\begin{tabular}{l l p{1.5in}}\textbf{Detail} & \textbf{Author} & \textbf{Description}\\\hline
	\href{https://github.com/NipunBathini/CS444/commit/9e822d032d044a83b1899a8fbf351415b689586d}{9e822d0} & Nipun & update test.c\\\hline
	\href{https://github.com/NipunBathini/CS444/commit/8691766dfe165c92fdc9c95d0d2ec4b9b948b3cd}{8691766} & Nipun & push patch\\\hline
	\href{https://github.com/NipunBathini/CS444/commit/df3b720ac50d7d541769061abdde739b3ed2e78d}{df3b720} & Nipun & push the new\\\hline
	\href{https://github.com/NipunBathini/CS444/commit/a69dec36b146a5b0567144f05d2407a4170d5d50}{a69dec3} & Nipun & Push old\\\hline
	\href{https://github.com/NipunBathini/CS444/commit/1913c18c7f6603c2633e8dd89c2664629f1534f6}{1913c18} & Nipun & Rename Design Plan.txt to HW4/Design Plan.txt\\\hline
	\href{https://github.com/NipunBathini/CS444/commit/4b6e10fb7e0f7a8ba3ca91a6eb70342902a2420d}{4b6e10f} & Nipun & Add files via upload\\\hline
	\end{tabular}

\section{Work log}

	11/26: began research on device memory allocation in linux
	
	11/26: began research on slob algorithms in linux
	
	11/26:  began research on syscalls in linux
	
	11/26: began latex write up file
	
	11/27: searched for linux slob.c best fit template files
	
	11/27: began work on slob.c file
	
	11/27: continued work on latex write-up
	
	11/28: continued work on slob.c
	
	11/28: continued work on latex write-up
	
	11/29: continued work on slob.c file
	
	11/29: modified syscall files and created syscall functions
	
	11/29: continued work on latex write-up
	
	11/29: testing
	
	11/30: continued ironing out slob.c algorithm
	
	12/1: continued work on latex write-up
	
	12/1: testing
	

\end{document}