\documentclass[onecolumn, draftclsnofoot,10pt, compsoc]{IEEEtran}
\usepackage{graphicx}
\usepackage{url}
\usepackage{setspace}

\usepackage{geometry}
\geometry{textheight=9.5in, textwidth=7in}
\usepackage{hyperref}

\def \DocType{		WriteUp
				}
			
\newcommand{\NameSigPair}[1]{\par
\makebox[2.75in][r]{#1} \hfil 	\makebox[3.25in]{\makebox[2.25in]{\hrulefill} \hfill		\makebox[.75in]{\hrulefill}}
\par\vspace{-12pt} \textit{\tiny\noindent
\makebox[2.75in]{} \hfil		\makebox[3.25in]{\makebox[2.25in][r]{Signature} \hfill	\makebox[.75in][r]{Date}}}}
% 3. If the document is not to be signed, uncomment the RENEWcommand below
\renewcommand{\NameSigPair}[1]{#1}




%group members: Nipun Bathini, Parker Bruni
%group number: 20




\begin{document}

\pagenumbering{arabic}


\section{Design Plan}

		We plan to implement encryption techniques that are already built into the linux kernel,
		specifically using linux/crypto.h library. We will simply use this libraries API
		to encrypt our data before we write to the device, and decrypt our data before we read from the
		device. 
	
\section{Questions}
	\begin{enumerate}
		\item \textbf{What do you think the main point of this assignment is?}
	
		We believe that the point of this assignment was to help us understand a linux kernel
		environment enough to create a block driver module and encrypt/decrypt data that is 
		transferred to and from it (I/O). This will help us understand the linux kernel and its
		functionality better so if we ever need to develop a device driver we will have the foundation
		to do so. 
		
		\item \textbf{How did you personally approach the problem? Design decisions, algorithm, etc.}
	
		We studied online about linux kernel drivers and modular drivers within the linux environment
		until we had a decent enough foundation to begin our project. We referenced online resources
		about linux device driver modules templates and added encryption methods and specifications
		to that template. We used the build in encryption library within linux "linux/crypto.h" to 
		encrypt and decrypt data from the virtual device as the data was written/read to/from the device.
		
		\item \textbf{How did you ensure your solution was correct? Testing details, for instance.}
			
		To ensure the solution was correct we utilized print statements within the code that
		would print the requested raw data before and after it was encrypted.
		This was to insure that the data was being properly encrypted before it was written to the device. 
		When the data was being read from the device our method was to print the encrypted cipher text of the 
		data to the terminal before it was decrypted and then print the raw data after it was decrypted.
		This was also to ensure that the data was being read and decrypted incorrectly. When testing our decryption
		via the key we simply changed the key within the code to use false key before decrypting, which would print garbage 
		data to the terminal. This proves that our key is being used for the correct encryption/decryption method.
		
		\item \textbf{What did you learn?}
	
		We learned more about the linux kernel and the linux OS as a whole and learned how to create
		modules within that environment. We learned about linux I/O encryption methods as well as methods for
		creating block device driver and communicating with a block device via the driver using the kernel. 
		
	\end{enumerate}

\section{Version Control Log}

\begin{tabular}{l l p{1.5in}}\textbf{Detail} & \textbf{Author} & \textbf{Description}\\\hline
	\href{https://github.com/NipunBathini/CS444/commit/1f3b2a036147a44d34f40c35604f5de969bdff76}{1f4b2a0} & Nipun & edit testing.txt\\\hline
	\href{https://github.com/NipunBathini/CS444/commit/faf83c302ee3c7cb1bab2586d302d8ff32c45602}{faf83c3} & Nipun & Added makefile and reorganized\\\hline
	\href{https://github.com/NipunBathini/CS444/commit/8787e9f738301c57976a08a0b3fadb9d79f1dfca}{8787e9f} & Nipun & Add patch file\\\hline
	\href{https://github.com/NipunBathini/CS444/commit/fe97fd648f38c2c9365bf1304f82c12b462905f4}{fe97fd6} & Nipun & First Add of File\\\hline
	\end{tabular}
	

\section{Work log}

	11/7: began research on device drivers/devices in linux
	
	11/7: began research on modules in linux
	
	11/7:  began research on I/O encryption methods in linux
	
	11/7: began latex write up file
	
	11/8: searched for block driver module templates
	
	11/8: searched for encryption method templates
	
	11/8: began work on .c file for the driver module
	
	11/8: continued work on latex write-up
	
	11/9: continued work on .c file for the driver module finished applying template moduel
	
	11/9: continued work on latex write-up
	
	11/10: continued work on .c file for the driver module added I/O encryption
	
	11/10: continued work on latex write-up
	
	11/10: testing
	
	11/11: continued work on .c file for the driver module added I/O encryption
	
	11/11: continued work on latex write-up
	
	11/11: testing
	

\end{document}