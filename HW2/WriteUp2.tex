\documentclass[onecolumn, draftclsnofoot,10pt, compsoc]{IEEEtran}
\usepackage{graphicx}
\usepackage{url}
\usepackage{setspace}

\usepackage{geometry}
\geometry{textheight=9.5in, textwidth=7in}
\usepackage{hyperref}

\def \DocType{		WriteUp
				}
			
\newcommand{\NameSigPair}[1]{\par
\makebox[2.75in][r]{#1} \hfil 	\makebox[3.25in]{\makebox[2.25in]{\hrulefill} \hfill		\makebox[.75in]{\hrulefill}}
\par\vspace{-12pt} \textit{\tiny\noindent
\makebox[2.75in]{} \hfil		\makebox[3.25in]{\makebox[2.25in][r]{Signature} \hfill	\makebox[.75in][r]{Date}}}}
% 3. If the document is not to be signed, uncomment the RENEWcommand below
\renewcommand{\NameSigPair}[1]{#1}
%%%%%%%%%%%%%%%%%%%%%%%%%%%%%%%%%%%%%%%%%%%%%%%%%%%%%%%%%%%%%%%%%%%%%%%%%%%%%%
%%%%%%%%%%%%%%%%%%%%%%%%%%%%%%%%%%%%%%%%%%%%%%%%%%%%%%%%%%%%%%%%%%%%%%%%%%%%%%
%group members: Nipun Bathini, Parker Bruni
%group number: 20
%%%%%%%%%%%%%%%%%%%%%%%%%%%%%%%%%%%%%%%%%%%%%%%%%%%%%%%%%%%%%%%%%%%%%%%%%%%%%%
%%%%%%%%%%%%%%%%%%%%%%%%%%%%%%%%%%%%%%%%%%%%%%%%%%%%%%%%%%%%%%%%%%%%%%%%%%%%%%
\begin{document}

\pagenumbering{arabic}


\section{Design Plan}
\section{Questions}
	\begin{enumerate}
		\item \textbf{What do you think the main point of this assignment is?}
	
		We believe that the point of this assignment was to allow us to learn the process of 
		task scheduling that is done by the kernel. This also allowed us to figure out various
		task scheduling algorithms that are implemented into the Linux kernel already. Knowing 
		how to select or create a good scheduling algorithm is good because it lets you have the 
		power to choose a scheduling algorithm that better suits the needs of an application or
		process. There is no perfect algorithm, many have pros and cons that are better or worse
		for specialized situations.
		
		\item \textbf{How did you personally approach the problem? Design decisions, algorithm, etc.}
	
		We looked for resources online and found out what it takes to allow a Linux system 
		to change the scheduling algorithm. We learned that you can either change the scheduler
		algorithm at run time or as a boot setting. For simplicities sake, we just change it
		at run time using the command "echo (algorithm of choice) > /sys/block/(disk)/queue/scheduler". We
		confirm that the algorithm has been changed with cat /sys/block/(disk)/queue/scheduler which displays 
		which algorithm is being used. For our scheduler we decided to go with the "CLOOK" approach, or circular look.
		This will dispatch requests for sector information in a linear fashion and wrap around to the other
		side of the disk when there are no more sector values higher than the current request sector value.
		We achieve this by sorting requests by sector value in the queue. The dispatched requests will flow
		chronologically up the queue and wrap around to the smallest sector value of the request queue after it dispatches the highest sector request. 
		
		\item \textbf{How did you ensure your solution was correct? Testing details, for instance.}
		
		Once we patched our linux default schedulers to include our sstf implementation we changed the scheduler using the make menuconfig command.
		Once we changed to our scheduler, we did a simple read and write test. We created a new "testFile" and echo'd various statements into that file.
		We then used "cat" to read out the file contents to the terminal to see if the contents matched the expected output from the echo commands and
		the file was structered as expected. From this, we can assume our algorithm is functioning correctly. We also used checked print statements 
		of our sorted list to make sure that it was indeed sorted and wrapping to the front of the list as expected. 
	
		\item \textbf{What did you learn?}
	
		We learned a lot about the details about various scheduling algorithms, the basic structure of what a scheduler
		does (merging and sorting), the levels of task completion from request to disk read and write, the process
		of choosing a scheduling algorithm from Linux, the structure of a hard disk drive, the pros and cons 
		of various scheduling algorithms, and more. 
		
	\end{enumerate}
	
\section{Version Control Log}
	\begin{tabular}{l l p{1.5in}}\textbf{Detail} & \textbf{Author} & \textbf{Description}\\\hline
	\href{https://github.com/NipunBathini/CS444/commit/1d482fea38c7a7daf2162699999aade0bc2e7057#diff-026641ace18fe3818f5340bdb0f502a0}{1d482fe} & Nipun & Add writeup\\\hline
	\href{https://github.com/NipunBathini/CS444/commit/bf0a15325a431b968b513684443627d0cf8b2a98#diff-026641ace18fe3818f5340bdb0f502a0}{bf0a153} & Nipun & Add folder\\\hline
	\end{tabular}


\section{Work log}

	10/26: began research on scheduling algorithms
	
	10/26: began research on structure of hard disk drives
	
	10/26: began research on levels of task completion (request to read/write on disk)
	
	10/27: began latex write up file
	
	10/27: began c file implementation
	
	10/27: edited kernel files to include our algorithm
	
	10/27: researched elevator algorithm implementations
	
	10/28: continued work on c file implementation
	
	10/28: continued work on latex write-up
	
	10/29: finished work on c file implementation
	
	10/29: created python testing script
	
	10/29: continued work on latex write-up
	
	10/30: testing via python scripts

\end{document}